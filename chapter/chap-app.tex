
\chapter{附录}
\label{chap:appendix}

附录是作为说明书(论文)的补充部分,并不是必需的。
\begin{enumerate}
\item{}下列内容可以作为附录编于说明书(论文)之后:
\begin{enumerate} \item{}为了说明书(论文)的完整,但编入正文又损于正文的处理和逻辑性,这一类材料包括比正文更为详细的信息研究方法和技术的途述,对于了解正文内容具有重要的补充意义;
\item{}由于篇幅过大或取材于复制品而不便编入正文的材料;
\item{}某些重要的原始数据、数学推导、计算程序、注释、框图、统计表、打印机输出样片、结构图等。
\end{enumerate}
\item{}
附录中的有关格式\\
  说明书(论文)的附录依次为“附录A”、“附录B”、“附录C”等编号。如果只有一个附录,也应编为“附录A”。\\
  附录中的图、表、公式的命名方法也采用上面提到的图、表、公式命名方法,只不过将章的序号换成附录的序号。
\end{enumerate}
